\documentclass[11pt]{amsart}



\addtolength{\textwidth}{.5cm}
\addtolength{\evensidemargin}{-2cm}
\addtolength{\oddsidemargin}{-2cm}
%\linespread{1}
\parskip .2cm

%\usepackage{eqparbox}
%\usepackage[numbers]{natbib}
%\renewcommand*{\bibnumfmt}[1]{\eqparbox[t]{bblnm}{[#1]}}







\usepackage[all]{xy}
\usepackage{url}
\usepackage{amssymb}
\usepackage{hyperref}
\usepackage[margin=1 in]{geometry}
\usepackage{color,soul}
\usepackage{amsthm}
\usepackage{lipsum}
% Theorem environments.
%

\usepackage{mathtools}
\DeclarePairedDelimiter{\ceil}{\lceil}{\rceil}
\DeclareMathOperator{\Spec}{Spec}
\DeclareMathOperator{\td}{tr.deg}
\DeclareMathOperator{\rr}{rat.rk}
\DeclareMathOperator{\Hom}{Hom}
\DeclareMathOperator{\coker}{coker}

\theoremstyle{plain}
\newtheorem{theorem}[subsubsection]{Theorem}
\newtheorem{proposition}[subsubsection]{Proposition}
\newtheorem{lemma}[subsubsection]{Lemma}
\newtheorem{Sublemma}[subsubsection]{Sublemma}
\newtheorem{corollary}[subsubsection]{Corollary}
\newtheorem{claim}[subsubsection]{Claim}

\theoremstyle{definition}
\newtheorem{definition}[subsubsection]{Definition}
\newtheorem{example}[subsubsection]{Example}
\newtheorem{examples}[subsubsection]{Examples}
\newtheorem{exercise}[subsubsection]{Exercise}
\newtheorem{situation}[subsubsection]{Situation}
\newtheorem{conjecture}[subsubsection]{Conjecture}
\newtheorem{notation}[subsubsection]{Notation}
\newtheorem{comment}[subsubsection]{Comment}
\newtheorem{remark}[subsubsection]{Remark}
\newtheorem{remarks}[subsubsection]{Remarks}
\newtheorem{setup}[subsubsection]{Setup}


\newtheorem{Theorem}[subsection]{Theorem}
\newtheorem{Proposition}[subsection]{Proposition}
\newtheorem{Lemma}[subsection]{Lemma}
\newtheorem{Corollary}[subsection]{Corollary}
\newtheorem{Claim}[subsection]{Claim}

\theoremstyle{definition}
\newtheorem{Definition}[subsection]{Definition}
\newtheorem{Example}[subsection]{Example}
\newtheorem{Exercise}[subsection]{Exercise}
\newtheorem{Situation}[subsection]{Situation}
\newtheorem{Conjecture}[subsection]{Conjecture}
\newtheorem{Notation}[subsection]{Notation}
\newtheorem{Comment}[subsection]{Comment}
\newtheorem{Remark}[subsection]{Remark}
\newtheorem{Remarks}[subsection]{Remarks}


\theoremstyle{plain}
\newtheorem{innercustomthm}{Theorem}
\newenvironment{customthm}[1]
  {\renewcommand\theinnercustomthm{#1}\innercustomthm}
  {\endinnercustomthm}

\theoremstyle{plain}  
\newtheorem{innercustomcor}{Corollary}
\newenvironment{customcor}[1]
  {\renewcommand\theinnercustomcor{#1}\innercustomcor}
  {\endinnercustomcor}
  
  



\numberwithin{equation}{subsubsection}
\numberwithin{equation}{subsubsection}
%\Euler for math | Palatino for rm | Helvetica for ss | Courier for tt
%\usepackage{euler} % math
%\usepackage{eulervm} % a better implementation of the euler package (not in gwTeX)
\normalfont
\usepackage[T1]{fontenc}{}
\usepackage{setspace}


\usepackage{tikz-cd}

\parskip .2cm
%\usepackage{biblatex} 
%\bibliography{Bibliography}



\makeatletter
\def\@tocline#1#2#3#4#5#6#7{\relax
  \ifnum #1>\c@tocdepth % then omit
  \else
    \par \addpenalty\@secpenalty\addvspace{#2}%
    \begingroup \hyphenpenalty\@M
    \@ifempty{#4}{%
      \@tempdima\csname r@tocindent\number#1\endcsname\relax
    }{%
      \@tempdima#4\relax
    }%
    \parindent\z@ \leftskip#3\relax \advance\leftskip\@tempdima\relax
    \rightskip\@pnumwidth plus4em \parfillskip-\@pnumwidth
    #5\leavevmode\hskip-\@tempdima
      \ifcase #1
       \or\or \hskip 1em \or \hskip 2em \else \hskip 3em \fi%
      #6\nobreak\relax
    \dotfill\hbox to\@pnumwidth{\@tocpagenum{#7}}\par
    \nobreak
    \endgroup
  \fi}
\makeatother
\usepackage{hyperref}





\pagestyle{plain}

\title{Completions and solidity}
\author{Rankeya Datta\vspace{-2ex}}
\address{}
\email{}
\thanks{}
%\date{April 11, 2017. \textcolor{red}{PRELIMINARY VERSION. DO NOT DISTRIBUTE}}                                           % Activate to display a given date or no date
%\subsection{}


\begin{document}
\maketitle

\section{Some facts on completions}
Let $I$ be an ideal of a ring $A$. By $\widehat{A}^I$, we mean the completion of $A$ with respect to the $I$-adic topology. Algebraically, 
$$\widehat{A}^I = \varprojlim_{n} A/I^n.$$
Note that $\widehat{A}^I$ is $I\widehat{A}^I$-adically separated (hence also $I$-adically separated as an $A$-module), since
$$\bigcap_{n \in \mathbb N} (I\widehat{A}^I)^n = \bigcap_{n \in \mathbb N} I^n\widehat{A}^I = (0).$$

\begin{lemma}
\label{completion-map-basic-properties}
The canonical map
$$i: A \rightarrow \widehat{A}^I$$
satisfies the following properties:
\begin{enumerate}
\item For all $n \in \mathbb{N}$, $\widehat{A}^I = i(A) + I^n\widehat{A}^I.$
\item If $A$ is Noetherian, then $i$ is a flat map. Moreover, $i$ is faithfully flat if $I$ is contained in the Jacobson radical of $A$.
\end{enumerate}
\end{lemma}

\begin{proof}
(1) follows from the inverse limit interpretation of $\widehat{A}^I$. For a proof of (2) see [Mat89, Theorem 8.8 and Theorem 8.14].
\end{proof}


\begin{corollary}
\label{completion-not-fg}
Let $A$ be a Noetherian ring and $I$ an ideal of $A$ which is contained in the Jacobson radical. If $A$ is not $I$-adically complete, then the canonical map
$$i: A \rightarrow \widehat{A}^I$$
is not finite.
\end{corollary}

\begin{proof}
Since $i$ is faithfully flat, when we tensor the short exact sequence
$$0 \rightarrow A \xrightarrow{i} \widehat{A}^I \rightarrow \coker(i) \rightarrow 0$$
by $A/I$, we retain a short exact sequence
$$0 \rightarrow A/I \rightarrow \widehat{A}^I/I\widehat{A}^I \rightarrow \coker(i)/I\coker(i) \rightarrow 0.$$
Since the induced map $A/I \rightarrow \widehat{A}^I/I\widehat{A}^I$ is an isomorphism, it follows that
$$\coker(i) = I\coker(i).$$
Thus if $i$ is finite, then $\coker(i)$ is a finitely generated $A$-module, and consequently by Nakayama's lemma,
$$\coker(i) = 0.$$
However, we assumed that $A$ is not $I$-adically complete, that is, $\coker(i) \neq 0$. Thus, $i$ cannot be finite.
\end{proof}

\begin{proposition}
\label{maps-to-I-adic-separated}
If $M$ is an $I$-adically separated $A$-module and $i: A \rightarrow \widehat{A}^I$ is the canonical map, then the induced map 
$$i^*: \Hom_A(\widehat{A}^I, M) \rightarrow \Hom_A(A, M)$$
is injective. 
\end{proposition}

\begin{proof}
Let $\varphi: \widehat{A}^I \rightarrow M$ be an $A$-linear map such that 
$$\varphi \circ i = 0.$$
It suffices to show that $\varphi = 0$. Let $x \in \widehat{A}$. By Lemma \ref{completion-map-basic-properties}, for any $n \in \mathbb{N}$, there exists $a_n \in A$ and $x_n \in I^n\widehat{A}^I$ such that
$$x = i(a_n) + x_n.$$
Then
$$\varphi(x) = \varphi(i(a_n)) + \varphi(x_n) = \varphi(x_n),$$
and $\varphi(x_n) \in I^nM$ by $A$-linearity. This shows that
$$\varphi(x) \in \bigcap_{n \in \mathbb N} I^nM = (0),$$
where the last equality follows because $M$ is $I$-adically separated. Thus $\varphi = 0$.
\end{proof}








\section{Completions and solidity}

Recall that an $A$-module $M$ is \textbf{solid} if $\Hom_A(M, A)$ is not the trivial module, that is, if there exists some non-trivial $A$-linear map $M \rightarrow A$. 

Our goal is to try to figure out when $\widehat{A}^I$ is a solid $A$-module.

\begin{proposition}
\label{blah}
Suppose $i: A \rightarrow \widehat{A}^I$ is the canonical map. For any 
$$\varphi \in \Hom_A(\widehat{A}^I, A),$$
we have
$$\varphi(1)\widehat{A}^I \subseteq i(A).$$
\end{proposition}

\begin{proof}
Consider the $A$-linear map 
$$i \circ \varphi: \widehat{A}^I \rightarrow \widehat{A}^I,$$
and let 
$$a \coloneqq \varphi(1).$$
We also have the $A$-linear map
$$\ell_a: \widehat{A}^I \rightarrow \widehat{A}^I,$$
given by left-multiplication by $a$. Now for any $x \in A$, 
$$\ell_a(i(x)) = ai(x) = i(ax) = i(\varphi(1)x) = i \circ \varphi (i(x)),$$
that is,
$$i^*(\ell_a) = i^*(i \circ \varphi),$$
where 
$$i^*: \Hom_A(\widehat{A}^I, \widehat{A}^I) \rightarrow \Hom_A(A, \widehat{A}^I)$$
is the map induced by pre-composition by $i$. Since $\widehat{A}^I$ is an $I$-adically separated $A$-module, it follows from Proposition \ref{maps-to-I-adic-separated} that $i^*$ is injective. Thus
$$\ell_a = i \circ \varphi,$$
and so,
$$\varphi(1)\widehat{A}^I = \ell_a(\widehat{A}^I) \subseteq i(A).$$
\end{proof}

{\vspace{2mm}}


\begin{corollary}
\label{completion-rarely-splits}
Let $I$ be an ideal of a ring $A$. If $A$ is not $I$-adically complete, then the canonical map
$$i: A \rightarrow \widehat{A}^I$$
does not split, that is, $i$ does not admit a left-inverse in $\textrm{Mod}_A$.
\end{corollary}

\begin{proof}
Suppose for contradiction that $\varphi$ is a splitting of $i$. Then $i$ is injective (since it admits a left-inverse) and
$$\varphi(1) = 1.$$
The previous proposition implies that
$$\widehat{A}^I = \varphi(1)\widehat{A}^I \subseteq i(A),$$
that is, $i$ is also surjective, which means $i$ is an isomorphism. However, we assumed $A$ is not $I$-adically complete!
\end{proof}

{\vspace{2mm}}

\begin{corollary}
\label{completions-usually-not-solid}
Let $A$ be a Noetherian ring and $I$ be an ideal of $A$ contained in the Jacobson radical. Suppose $\varphi \in \Hom_A(\widehat{A}^I, A)$. Then we have the following:
\begin{enumerate}
	\item $\varphi \neq 0$ if and only if $\varphi(1) \neq 0$.
	\item If $A$ is not $I$-adically complete and $\varphi \neq 0$, then $\varphi(1)$ is a zerodivisor of A.
	\item If $A$ is a domain that is not $I$-adically complete, then $\Hom_A(\widehat{A}^I, A) = 0$. That is, $\widehat{A}^I$ is not a solid $A$-algebra.
\end{enumerate}
\end{corollary}


\begin{proof}
(3) clearly follows from (1) and (2), and (1) follows from the injectivity of the induced map
$$i^*: \Hom_A(\widehat{A}^I, A) \rightarrow \Hom_A(A,A),$$
where $i: A \rightarrow \widehat{A}^I$ is the canonical map. Note that $i^*$ is injective by Proposition \ref{maps-to-I-adic-separated}, since $A$ is $I$-adically separated by Krull's intersection theorem. 

Finally, for (2), if $\varphi  \neq 0$, then by (1), 
$$\varphi(1) \neq 0.$$ 
Assume that $\varphi(1)$ is a non-zerodivisor of $A$. Then by flatness of $i: A \rightarrow \widehat{A}^I$, $\varphi(1)$ is also a non-zerodivisor of $\widehat{A}^I$. In particular,
$$\varphi(1)\widehat{A}^I \cong \widehat{A}^I.$$
At the same time, Proposition \ref{blah} shows that
$$\varphi(1)\widehat{A}^I \subseteq i(A),$$
and so $\varphi(1)\widehat{A}^I$ is a finitely generated $A$-module, since it is a submodule of the finitely generated $A$-module $i(A)$ and $A$ is Noetherian. Thus $\widehat{A}^I$ is a finitely generated $A$-module, which contradicts the hypothesis that $A$ is not $I$-adically complete by Corollary \ref{completion-not-fg}.
\end{proof}

\vspace{5mm}

\textcolor{red}{Question: Does part (3) of the previous corollary have counter-examples when $A$ is not a domain?}


\end{document}
